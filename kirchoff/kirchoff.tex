\documentclass{article}

\usepackage{amsmath}
\usepackage{cancel}

\title{Matemática IV - Transformada de Laplace, uso en Teoría de Control}
\author{
    Leguina, Damián Adolfo
    \and
    Jorge Michelle
    \and
    Estaban Ronaldo
}
\date{\today}

\begin{document}

\section*{Ley de Kirchoff}

\[
    L \frac{d I}{d t} + RI + \frac{Q}{C} = E 
    \qquad\qquad
    I = \frac{d Q}{d t}
    \Rightarrow 
    \boxed{
        I' = \frac{d^2Q}{d t^2}
    } = \frac{d I}{ d t}
\]

\begin{align*}
    2 \frac{d^2 Q}{d t^2}
    + 16 \frac{d Q}{d t}
    + \frac{1}{0,02} Q
    &= 300 V \\
    \frac{
        2 \frac{d^2 Q}{d t^2}
        + 16 \frac{d Q}{d t} + 50 Q
        }{2}
    &= \frac{300 V}{2} \\
    \frac{d^2 Q}{d t^2}
    + 8 \frac{d Q}{d t}
    + 25 Q
    &= 150 V
\end{align*}


\section*{Aplicando la transformada}

\begin{align*}
    \mathcal{L}\left[\frac{d^2 Q}{d t^2}\right]
    + 8 \mathcal{L}\left[\frac{d Q}{d t}\right]
    + 25 \mathcal{L}\left[Q\right]
    &= 150 V \mathcal{L}\left[1\right] \\
    s^2 Q\left(s\right)
    - \cancelto{0}{Q\left(0\right)}
    - \cancelto{0}{Q'\left(0\right)}
    + 8 \left[sQ\left(s\right) - \cancelto{0}{Q\left(0\right)}\right]
    + 25 Q \left(s\right)
    &= \frac{150 V}{s} \\
    Q(s)\left[s^2 + 8 s + 2 s\right] &= \frac{150 V}{s}
\end{align*}

\begin{align*}
    Q\left(s\right)\left[s^2 + 8s + 25\right] &= \frac{150v}{s} \\
    Q\left(s\right) &= \frac{150v}{s\left(s^2 + 8s + 25\right)} \\
    &= \frac{A}{s} + \frac{Bs + C}{s^2 + 8s + 25}
\end{align*}

\begin{align*}
    150V &= A \left(s^2 + 8s + 25\right) + \left(Bs + c\right) s \\
    &= As^2 + A8s + A25 + Bs^2 + Cs
\end{align*}

\begin{align*}
    25A &= 150 \\
    A &= \frac{150}{25} \rightarrow \boxed{A = 6} \\
    8A + C &= 0 \\
    C &= -8A \\
    &= -8 \times 6 \rightarrow \boxed{C = -48} \\
    A + B &= 0 \\
    B &= -A \rightarrow \boxed{B = -6}
\end{align*}

\begin{align*}
    Q\left(s\right) &= \frac{6}{s} - \frac{6s + 48}{s^2 + 8s + 25} = \frac{6}{s} - \frac{6s + 48}{\left(s+4\right)^2 + 9} \\
    \Downarrow \\
    Q\left(t\right) &= \mathcal{L}^-1 \left[\frac{6}{s} - \frac{6s + 48}{\left(s+4\right)^2 + 9}\right] \\
\end{align*}

\begin{align*}
    \boxed{
        Q\left(t\right) = 6 - 6 e^{4t} cos\left(3t\right) - 8 e^{4t} sen \left(3t\right)
    }
\end{align*}

\begin{align*}
    I = \frac{d Q}{d t} &= \left[6 - 6 e^{4t} cos\left(3t\right) - 8 e^{4t} sen \left(3t\right)\right] \frac{d}{dt} \\
    &= 6 \frac{d}{dt} - 6 \left[e^{4t} cos\left(3t\right) \frac{d}{dt}\right] - 8 \left[e^{4t} sen \left(3t\right) \frac{d}{dt}\right] \\
    %&= 0 -6 e^{4t} \left[4 cos \left(3t\right) - 3 sen\left(3t\right)\right] -8 e^{4t} \left[4 sen \left(3t\right) - 3 cos\left(3t\right)\right]
\end{align*}

\end{document}
% \mathcal{L}^-1 \left[\right]
% \[
%     \mathcal{L}\{f(t)\}=\int_{t=0}^{\infty}f(t)e^{-st}dt
% \]