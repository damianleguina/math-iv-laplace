\documentclass{article}

\usepackage{amsmath}
\usepackage{amssymb}
\usepackage{graphicx}
\usepackage{cancel}
\usepackage{hyperref}

\hypersetup{
    colorlinks=true,
    linkcolor=black,
    filecolor=black,      
    urlcolor=blue,
}
\graphicspath{ {./images/} }

\title{Matemática IV - Transformada de Laplace, uso en Ingeniería Mecánica}
\author{
    Juan Ignacio Agüero, Damián Adolfo Leguina y Paulina Antico
}
\date{\today}

\begin{document}

\maketitle

\tableofcontents

\newpage

\section{Transformada de Laplace}

Nombrada así en honor al matemático \emph{Pierre-Simon de Laplace} (1749-1827), es una transformada integral que convierte una función de una variable real, (usualmente \(t\)), al dominio algebraico \(s\).

\subsection*{Formula}

Establezcamos \(f(t)\) para \(0 \leq t \le \infty\) y \(s\) denotando una variable real arbitraria. La \emph{transformada de Laplace} de \(f(x)\) designada o bien por \(\mathcal{L}\{f(t)\}\) o bien \(F(s)\), es

\begin{align*}
    \mathcal{L}\{f(t)\}=\int_{t=0}^{\infty}e^{-st}f(t)dt
\end{align*}

\newpage

\section{Transformada de Laplace Inversa}

Es la función inversa a la Transformada de Laplace, representada componentes

\begin{align*}
    \mathcal{L}^{-1}\left[F\left(s\right)\right](t)
\end{align*}

La transformada inversa de Laplace permita transformar una función en el dominio algebraico al dominio real.

\subsection*{Formula}

Existen varias formulas para representar la transformada inversa de Laplace, una de ellas es la conocida como la Formula Inversa de Mellin, también llamada integral de Bromwich o integral de Fourier-Mellin:

\begin{align*}
    f\left(t\right) = \mathcal{L}^{-1} \left\{F\left(s\right)\right\} = \frac{1}{2\pi i} \lim_{T \to +\infty} \int_{\gamma - iT}^{{\gamma + iT}} e^{st} F \left(s\right) ds
\end{align*}

\newpage

\section{Ingeniería Mecánica}

La \emph{Ingeniería Mecánica} es una rama de la ingeniería que combina ingeniería física con matemática para poder diseñar, analizar, crear y mantener sistemas mecánicos.

\subsection{relación con la transformada de Laplace}


En la Ingeniería Mecánica, la \emph{transformada de Laplace} se usa para resolver ecuaciones diferenciales de un sistema mecánico para encontrar la función de transferencia\footnote{La \emph{función de transferencia} de un sistema es una función que modela la salida de un sistema para cada entrada posible.} de ese sistema.

\subsection{Ejemplo: Tanque de agua}

\includegraphics{tank}

La imagen muestra un tanque de agua llenándose a un radio $f_{entrada}$, y se vacía a un radio $f_{salida}$, en el instante $t = 0$, el tanque esta vacío. Lo primero que podemos deducir de este sistema es que va a haber una tasa relacionando $f_{entrada}$ con $f_{salida}$, sí $f_e = f_s$, entonces la altura del agua acumulada $h$ permanece constante, si $f_e > f_s$, $h$ va a crecer, y $f_e < f_s$, $h$ decrece.

\subsubsection*{Ecuación diferencial}

En este caso, podemos asumir que el agua acumulada es el cambio del volumen del agua en el tiempo.

\begin{align*}
    f_e - f_s = \frac{\Delta V}{\Delta t}
\end{align*}

El volumen siendo el area $A$ por la altura, asumiendo que el area de la sección transversal es constante, la altura $h$ es la única variable en la función del volumen, por lo que la variación del volumen $\Delta V = A \cdot d h$, por lo que la ecuación diferencial queda en:

\begin{align*}
    f_e - f_s = \frac{A \cdot d h}{d t} = A \frac{d h}{d t}
\end{align*}

\subsubsection*{Flujo de salida}

El flujo de salida esta dominado por la presión $p$ que ejerce el agua para salir por el tubo que funciona como una restricción $R$, por lo tanto $f_s = p/R$. La presión esta definida como $p = \rho gh$, donde $\rho$ es la densidad del agua, $h$ es la altura del cuerpo de agua en un moment $t$ y $g$ la aceleración de la gravedad, por lo tanto

\begin{align*}
    f_s = \frac{\rho gh}{R}
\end{align*}

Reemplazando en la ecuación diferencial se puede escribir como:

\begin{align*}
    f_e &- \frac{\rho gh}{R} = A \frac{dh}{dt}\\
    \therefore f_e &= A \frac{dh}{dt} + \frac{\rho gh}{R}\\
\end{align*}

\subsubsection*{Aplicando Laplace}

\begin{align*}
    \mathcal{L}\left[f_e\right]\left(s\right) &= \mathcal{L}\left[A \frac{dh}{dt} + \frac{\rho gh}{R}\right]\left(s\right) \\
    \therefore F_e\left(s\right) &= AsH\left(s\right) - \cancelto{0}{h\left(0\right)} + \frac{\rho g}{R}H\left(s\right) \\
    \therefore F_e\left(s\right) &= AsH\left(s\right) + \frac{\rho g}{R}H\left(s\right) \\
    \therefore \label{eq:3} F_e\left(s\right) &= \left[As + \frac{\rho g}{R}\right]H\left(s\right)
\end{align*}

\subsubsection*{Función de transferencia}

La función transferencia $G(s)$ se determina con la expresión

\begin{align*}
    G\left(s\right) = \frac{Y\left(s\right)}{X\left(s\right)}
\end{align*}

Donde $X\left(s\right)$ es la entrada de un sistema e $Y\left(s\right)$ la salida.
En nuestro sistema, es $F_e\left(s\right)$, la salida $H\left(s\right)$, por lo que tenemos que adaptar nuestra ecuación.
\begin{align*}
    F_e\left(s\right) &= \left[As + \frac{\rho g}{R}\right]H\left(s\right) \\
    \frac{F_e\left(s\right)}{As + \frac{\rho g}{R}} &= H\left(s\right) \\
    \frac{1}{As + \frac{\rho g}{R}} &= \frac{H\left(s\right)}{F_e\left(s\right)} \Rightarrow \boxed{G\left(s\right) = \frac{1}{As + \frac{\rho g}{R}}}
\end{align*}

De esta manera obtenemos obtuvimos la función de transferencia, la principal ventaja de una función de transferencia es la de poder minimizar los componentes de un sistema a ecuaciones algebraicas simples en lugar de utilizar ecuaciones diferenciales complejas para el análisis y diseño de sistemas.

\subsubsection*{Transformada inversa de Laplace y función transferencia}

El uso de la transformada de Laplace en una función transferencia es util para saber cual es el valor de la salida en un momento $y(t)$, dado que:

\begin{align*}
    G\left(s\right) &= \frac{Y\left(s\right)}{X\left(s\right)}\\
    Y\left(s\right) &= G\left(s\right)X\left(s\right)\\
    y\left(t\right) = \mathcal{L}^{-1}\left[Y\left(s\right)\right] &= \mathcal{L}^{-1} \left[G\left(s\right)X\left(s\right)\right]\\
\end{align*}

Para simplificar, vamos a imaginar que $f_e$ es una entrada tipo impulso unitario, lo que significaría que $F\left(s\right) = 1$.

\begin{align*}
    H\left(s\right) &= G\left(s\right) \cdot F_e\left(s\right) \\
    H\left(s\right) &= \frac{1}{As + \frac{\rho g}{R}} \cdot 1 \\
    H\left(s\right) &= \frac{1}{As + \frac{\rho g}{R}} \cdot \frac{\frac{1}{A}}{\frac{1}{A}} \\
    H\left(s\right) &= \frac{\frac{1}{A}}{s + \frac{\rho g}{AR}} \\
    \mathcal{L}^{-1}\left[H\left(s\right)\right] &= \mathcal{L}^{-1}\left[\frac{\frac{1}{A}}{s + \frac{\rho g}{AR}}\right] = \frac{1}{A}\mathcal{L}^{-1}\left[\frac{1}{s + \frac{\rho g}{AR}}\right]\\
    h\left(t\right) &= \frac{1}{A} \cdot e^{-\frac{\rho g}{AR} t} \\
\end{align*}

\newpage

\section{Bibliografía}

\begin{description}
    \item[ECE 209: Review of Circuits as LTI\dots] - Theodore P. Pavlic \href{https://www.tedpavlic.com/teaching/osu/ece209/support/circuits_sys_review.pdf}{[Link]}
    \item[APPLICATIONS OF LAPLACE\dots] - Prof. L.S. Sawant \href{https://www.irjet.net/archives/V5/i5/IRJET-V5I5593.pdf}{[Link]}
    \item[Modeling a Process - Filling a Tank] - Gavin Duffy \href{https://eleceng.dit.ie/gavin/Control/Modeling/Filling%20a%20Tank.htm}{[Link]}
    \item[Transfer Functions (Continuous-Time)] - Dr James E. Pickering \href{https://www.youtube.com/playlist?list=PLWOqcVY--6BeiKryG_5hw0TCAsCde1sBL}{[Link]}
 \end{description}]

\end{document}